\documentclass{beamer}

\usetheme{m}
\usepackage{FiraSans}
\usepackage[utf8]{inputenc}
\usepackage[british]{babel}

\title{Observing Arctic sea ice with Perl}
\author{Paul Cochrane}

\begin{document}

\begin{frame}
    \titlepage
\end{frame}

my name's Paul, I'm currently a freelance software developer and I just
recently stopped working for this company (it's a start up and funding ran
out for devs, however funding is upcoming).

\begin{frame}
    DNPS logo large
\end{frame}

\begin{frame}
    But that's not important right now
\end{frame}

what *is* important, is that the company focusses on sea ice, and as part of
running the company's social network presence (I'm a jack of all trades), I
came across a talk by David Wasdell on our Twitter feed about Arctic
feedback dynamics and the melting of Arctic sea ice.  I got to 11:22 minutes
and noticed something which didn't match current observations.

\begin{frame}
Arctic Feedback Dynamics Presentation by David Wasdell (Part 1)
https://www.youtube.com/watch?v=AjZaFjXfLec
at about 11:22 mentions exponential extrapolation and that 2015 all ice will
disappear in summer.  The funny thing is, it does't look like it (doesn't
really fit reality).

- "prediction" via extrapolation -> exponential fit

His talk is actually from 2013, with data from 2012 (dates correct?) so it's not too bad,
however the conclusion that Arctic sea ice will disappear in 2015 doesn't
look like it's going to happen.

sea ice extent data is available for free online, back to the 70's (NSIDC).  I can
run the investigation similarly myself...
\end{frame}

first quick look at data with a spreadsheet, however that's not fun, and I
want to be able to run the program more often and maybe fiddle with things a
bit more than I can with a spreadsheet.  So, I pulled out my favourite swiss
army chainsaw.

- grab data (LWP::Simple)
- hey, it's CSV (Text::CSV)
- comes in two files, "archive" and "near real time (nrt)"
- stitch them together
- extract minima for a given year
- plot minima (Gnuplot bindings???, maybe SVG::Plot???)
- fit library?  Linear regression from scratch?
- what is the equation for the the fit?  (Python module, since there isn't a
Perl module)
- high school maths -> solve for y=0
- can one reproduce the old result from the video?
- what does the extrapolation tell us?
- mention technologies/modules used

\begin{frame}
linear is obviously wrong
exponential doesn't give as good a fit as 2nd order polynomial
\end{frame}

\begin{frame}
    \frametitle{Think!}
    \begin{itemize}
	\item Don't believe everything you read in blogs
	\item Don't believe every video you see online
	\item FFS don't believe me!
	\item Critically evalutate everything
	\item The tools are in \emph{your} hands, the data is available for
	    free, can come to own conclusion.
	    \begin{itemize}
		\item Perl, CPAN, freely available data, your brain
	    \end{itemize}
    \end{itemize}
\end{frame}




% - working for DNPS
% - did not only development, but also ran social media stuff
% - in Twitter feed came across video about Arctic feedback dynamics
% - video interesting and informative, however ...
% - makes a prognosis about Arctic sea ice disappear in summer 2015
% - uses data up to 2011 and sort of 2012, was uploaded on 2013, I saw the
%   video in 2014 and, well, it's taken longer to prepare this talk than I
%   would have liked
% - Arctic sea ice wasn't gone in the northern summer of 2015, so... wtf?
% - wtf is how one writes hä in English
% - the data is freely available online (e.g. NSIDC), so why not check the
%   results? (btw, verification of results is an important part of science)
% - let's look at the presented results carefully: a linear fit of the
%   1979-2011 data gives a zero crossing in 2075; "using the curves" and
%   data from what looks like 1995-2012 (data point at 2012 looks too low),
%   gives a zero crossing at 2015
% - where do I get the data?
% - it's in an archived form, and as near-real-time
% - stich them together, plot
% - we really want the minima, extract them
% - what's the linear fit?
% - what's the zero crossing of the linear fit?  (roughly 2073, so given
%   errors a verification of result presented in talk)
% - talk uses a fit-data-and-extrapolate approach; we can do that too
% - a second order polynomial fits quite well


% - take away message: don't believe everything!  Think critically!  Use
%   your brain and the tools at hand.  The tools and data are often freely
%   available.

\end{document}
