\documentclass{beamer}

\usetheme{m}
\usepackage{FiraSans}
\usepackage[utf8]{inputenc}
\usepackage[british]{babel}

\title{Observing Arctic sea ice with Perl}
\author{Paul Cochrane}

\begin{document}

\begin{frame}
    \titlepage
\end{frame}

my name's Paul, I'm currently a freelance software developer and I just
recently stopped working for this company (it's a start up and funding ran
out for devs, however funding is upcoming).

\begin{frame}
    DNPS logo large
\end{frame}

\begin{frame}
    But that's not important right now
\end{frame}

what *is* important, is that the company focusses on sea ice, and as part of
running the company's social network presence (I'm a jack of all trades), I
came across a talk by David Wasdell on our Twitter feed about Arctic
feedback dynamics and the melting of Arctic sea ice.  I got to 11:22 minutes
and noticed something which didn't match current observations.

\begin{frame}
Arctic Feedback Dynamics Presentation by David Wasdell (Part 1)
https://www.youtube.com/watch?v=AjZaFjXfLec
at about 11:22 mentions exponential extrapolation and that 2015 all ice will
disappear in summer.  The funny thing is, it does't look like it.

His talk is actually from 2013, with data from 2012 so it's not too bad,
however the conclusion that Arctic sea ice will disappear in 2015 doesn't
look like it's going to happen.

sea ice extent data is available for free online, back to the 70's.  I can
run the investigation similarly myself...
\end{frame}

first quick look at data with a spreadsheet, however that's not fun, and I
want to be able to run the program more often and maybe fiddle with things a
bit more than I can with a spreadsheet.  So, I pulled out my favourite swiss
army chainsaw.

- grab data (LWP::Simple)
- hey, it's CSV (Text::CSV)
- comes in two files, "archive" and "near real time (nrt)"
- stitch them together
- extract minima for a given year
- plot minima (Gnuplot bindings???, maybe SVG::Plot???)
- fit library?  Linear regression from scratch?
- what is the equation for the the fit?
- what does the extrapolation tell us?

\begin{frame}
linear is obviously wrong
exponential doesn't give as good a fit as 2nd order polynomial
\end{frame}

\begin{frame}
    \frametitle{Think!}
    \begin{itemize}
	\item Don't believe everything you read in blogs
	\item Don't believe every video you see online
	\item FFS don't believe me!
	\item The tools are in \emph{your} hands
	    \begin{itemize}
		\item Perl, CPAN, freely available data, your brain
	    \end{itemize}
    \end{itemize}
\end{frame}


% XXX: "I want to believe" poster crossed out?

\end{document}
